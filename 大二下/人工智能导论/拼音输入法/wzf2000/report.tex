\documentclass[a4paper]{article}
\usepackage{graphicx}
\usepackage{epstopdf}
\usepackage{placeins}
\usepackage{amsmath}
\usepackage{array}
\usepackage{ctex}
\usepackage{geometry}
\usepackage{mdwlist}
\usepackage{amsbsy}
\usepackage{amssymb}
\usepackage{enumitem}
\usepackage{tikz}
\usepackage{xcolor}
\usepackage{listings}
\usepackage{multirow}
\usepackage{booktabs}
\usepackage{float}
\usepackage{framed}
\usepackage{color}
\definecolor{lightgray}{rgb}{0.75,0.75,0.75}
\newenvironment{lightgrayleftbar}{%
  \def\FrameCommand{\textcolor{lightgray}{\vrule width 3pt} \hspace{3pt}}%
  \MakeFramed {\advance\hsize-\width \FrameRestore}}%
{\endMakeFramed}
\usetikzlibrary{trees}
\geometry{a4paper,scale=0.75}
\setlist[description]{leftmargin=*}
\setmainfont{Times New Roman}
\lstdefinestyle{lfonts}{
    basicstyle   = \footnotesize\ttfamily,
    stringstyle  = \color{purple},
    keywordstyle = \color{blue!60!black}\bfseries,
    commentstyle = \color{olive}\scshape,
}
\lstdefinestyle{lnumbers}{
    numbers     = left,
    numberstyle = \tiny,
    numbersep   = 1em,
    firstnumber = 1,
    stepnumber  = 1,
}
\lstdefinestyle{llayout}{
    breaklines       = true,
    tabsize          = 2,
    columns          = flexible,
}
\lstdefinestyle{lgeometry}{
    xleftmargin      = 20pt,
    xrightmargin     = 0pt,
    frame            = tb,
    framesep         = \fboxsep,
    framexleftmargin = 20pt,
}
\lstdefinestyle{lgeneral}{
    style = lfonts,
    style = lnumbers,
    style = llayout,
    style = lgeometry,
}
\lstdefinestyle{python}{
    language = {Python},
    style    = lgeneral,
}
\title{人工智能导论------拼音输入法实验报告}
\author{计 93 王哲凡 2019011200}

\date{\today}

\begin{document}
    \maketitle
    \tableofcontents

    \newpage
    \section{算法介绍}

    本次实验我主要采用了字的二元模型和三元模型算法。

    对于每个潜在的汉字选项,我们给出了其对应的评价函数。

    \subsection{数据统计}

    首先遍历给定的 sina 新闻数据集,提取其中的 title 和 html 部分,将每个部分作为一个潜在的学习语料,此部分在 \texttt{src/datareader.py} 中的 \texttt{DataReader} 类中实现。 

    然后我们通过 \texttt{pypinyin} 库对语料进行预处理的拼音标注,此处采用了库中的 \texttt{lazy\_pinyin()} 方法,主要用以处理语料中多音字的具体读音情况。

    结合通过上面方法标注的拼音,我们遍历所有 sina 语料,统计了其中各个汉字字的出现次数(记录在 \texttt{dataset/frequency-1.json} 下),以及相邻二字组成的不同二元组出现次数(记录在 \texttt{dataset/frequency-2.json} 下),对于字三元模型,则进一步统计了相邻三个汉字构成的三元组的出现次数(记录在 \texttt{dataset/frequency-3.json} 下),这部分处理均由 \texttt{src/preprocess.py} 中的函数完成。

    除此之外,我还统计了单拼音对应汉字的字典 \texttt{dataset/word\_table.json} 和单汉字对应拼音的字典 \texttt{dataset/pinyin\_table.py}。

    \texttt{dataset} 文件夹下 \texttt{sentences-XX.json} 为 sina 语料的预处理中间文件,并未被模型使用,而 \texttt{word\_frequency-X.json} 则为通过 \texttt{jieba} 库分词后的词频率统计,目前暂未使用。

    \subsection{字二元模型}

    对给定的一个拼音序列 \(\{p_1, p_2, \cdots, p_n\}\),其中 \(p_i\) 为其中第 \(i\) 个汉字的拼音,我们希望得到的是在给定这个拼音序列条件下,最高概率的汉字序列(即句子),即:

    \[W_0 = \mathop{\arg \max}\limits_{w_1, w_2, \cdots, w_n} P(w_1 w_2 \cdots w_n | p_1 p_2 \cdots p_n)\]

    根据贝叶斯公式:

    \[P(w_1 w_2 \cdots w_n | p_1 p_2 \cdots p_n) = \frac{P(p_1 p_2 \cdots p_n | w_1 w_2 \cdots w_n) \cdot P(w_1 w_2 \cdots w_n)}{P(p_1p_2 \cdots p_n)}\]

    \begin{lightgrayleftbar}
        \textbf{贝叶斯公式}:
        \[P(X | Y) = \frac{P(Y | X) P(X)}{P(Y)}\]
    \end{lightgrayleftbar}

    由于 \(\{p_1, \cdots, p_n\}\) 序列确定,故 \(P(p_1 p_2 \cdots p_n)\) 为常数,而 \(P(p_1 p_2 \cdots p_n | w_1 w_2 \cdots w_n)\) 可认为是 \(1\)(通过将字拆分成对应读音的分字可保证其为 \(1\)),因此我们只须关注 \(P(w_1 w_2 \cdots w_n) = P(W)\)。

    \newpage
    通过条件概率可化简为:

    \begin{align*}
        W_0 & = \mathop{\arg \max}\limits_{w_1, w_2, \cdots, w_n} P(W) \\
        & = \mathop{\arg \max}\limits_{w_1, w_2, \cdots, w_n} P(w_1 w_2 \cdots w_n) \\
        & = \mathop{\arg \max}\limits_{w_1, w_2, \cdots, w_n} P(w_1) P(w_2 |w_1) P(w_3 | w_1 w_2) \cdots P(w_n | w_1 w_2 \cdots w_{n - 1}) \\
        & \approx \mathop{\arg \max}\limits_{w_1, w_2, \cdots, w_n} P(w_1) P(w_2 | w_1) P(w_3 | w_2) \cdots P(w_n | w_{n - 1}) \\
        & = \prod_{i = 1}^n P(w_i | w_{i - 1})
    \end{align*}

    其中 \(w_i\) 表示第 \(i\) 个汉字的预测,\(P(w_i | w_1 w_2 \cdots w_{i - 1})\) 表示前 \(i - 1\) 个字分别确定为 \(w_1, \cdots, w_{i - 1}\) 时,第 \(i\) 个字为 \(w_i\) 的概率,\(P(w_i | w_{i - 1})\) 表示前一个字为 \(w_{i - 1}\) 时,后一个字为 \(w_i\) 的概率(在给定拼音条件下),此处默认 \(P(w_1 | w_0) = P(w_1)\) 即第一个拼音对应 \(w_1\) 的概率。

    又根据条件概率公式:

    \[P(w_i | w_{i - 1}) = \frac{P(w_{i - 1} w_i)}{P(w_{i - 1})}\]

    其中 \(P(w_{i - 1} w_i)\) 即表示 \(w_{i - 1}\) 与 \(w_i\) 相邻出现的概率,\(P(w_{i - 1})\) 表示 \(w_{i - 1}\) 出现的概率。

    根据大数定律,我们可以用语料中的统计频率来逼近估计,即:

    \[P(w_i | w_{i - 1}) \approx \frac{\mathrm{count}(w_{i - 1} w_i)}{\mathrm{count}(w_{i - 1})} = P^*(w_i | w_{i - 1})\]

    其中 \(\mathrm{count}(w_{i - 1} w_i), \mathrm{count}(w_{i - 1})\) 分别表示语料中 \(w_{i - 1} w_i\) 和 \(w_i\) 出现的频次。

    而考虑到部分 \(P(w_{i - 1} w_i)\) 较低的情况,可能导致在语料库中 \(\mathrm{count}(w_{i - 1} w_i)\) 退化为 \(0\),因此通过 laplace 平滑可将上式改写为:

    \[P(w_i | w_{i - 1}) \approx \alpha P^*(w_i | w_{i - 1}) + (1 - \alpha) P^*(w_i), \alpha \approx 1\]

    其中 \(P(w_i)\) 的估计 \(P^*(w_i)\) 为:

    \[P^*(w_i) = \frac{\mathrm{count}(w_i)}{r}\]

    \begin{lightgrayleftbar}
        模型中统一取 \(r = 100000\)。
    \end{lightgrayleftbar}

    \subsection{求模型最优解}

    模型文件为 \texttt{src/models.py},其中构造函数实现了读入语料数据,\texttt{forward()} 函数实现了最优预测。

    模型中的下标为 \(0, \cdots, n - 1\),与下面的规定略有不同。

    设 \(\mathrm{dp}_{i, w}\) 表示前 \(i\) 个汉字,第 \(i\) 个为 \(w\) 的概率对数最大值,\(\mathrm{last}_{i, w}\) 则表示取得最大值情况下选择的第 \(i - 1\) 个汉字。

    初始值为:

    \[\mathrm{dp}_{1, w} = \log P(w), w \in \mathrm{Word}(p_1)\]

    其中 \(\mathrm{Word}(p)\) 表示拼音 \(p\) 所对应的汉字集合。

    容易得到转移方程:

    $$
    \begin{cases}
    \mathrm{dp}_{i + 1, w} = \max\limits_{w' \in \mathrm{Word}(p_i)} \left(\mathrm{dp}_{i, w'} + \log P(w | w') \right) \\
    \mathrm{last}_{i + 1, w} = \mathop{\arg \max}\limits_{w' \in \mathrm{Word}(p_i)} \left(\mathrm{dp}_{i, w'} + \log P(w | w') \right)
    \end{cases}
    $$

    通过动态规划容易求解得到完整的 \(\mathrm{dp}\) 和 \(\mathrm{last}\) 结果,再通过 \(\mathrm{last}\) 数组倒推即可得到完整的预测汉字序列。

    \begin{lightgrayleftbar}
        此处的 \(P\) 均为 1.2 模型中给出的语料统计估计值。
    \end{lightgrayleftbar}

    \subsection{字三元模型}

    通过单个汉字预测下一个汉字的二元模型有较大的的局限性,考虑通过前两个汉字来分析下一个汉字,即:

    \begin{align*}
        W_0 & = \mathop{\arg \max}\limits_{w_1, w_2, \cdots, w_n} P(w_1) P(w_2 |w_1) P(w_3 | w_1 w_2) \cdots P(w_n | w_1 w_2 \cdots w_{n - 1}) \\
        & \approx \mathop{\arg \max}\limits_{w_1, w_2, \cdots, w_n} P(w_1) P(w_2 | w_1) P(w_3 | w_1 w_2) \cdots P(w_n | w_{n - 2} w_{n - 1}) \\
        & = \prod_{i = 1}^n P(w_i | w_{i - 2} w_{i - 1})
    \end{align*}

    同样需要考虑退化情况,处理为:

    \begin{align*}
        P(w_i | w_{i - 2} w_{i - 1}) & \approx \beta P^*(w_i | w_{i - 2} w_{i - 1}) + (1 - \beta) P(w_i | w_{i - 1}) & \beta \approx 1 \\
        & \approx \beta P^*(w_i | w_{i - 2} w_{i - 1}) + (1 - \beta) \left[\alpha P^*(w_i | w_{i - 1}) + (1 - \alpha) P^*(w_i) \right] & \alpha \approx 1
    \end{align*}

    其中:

    \[P^*(w_i | w_{i - 2} w_{i - 1}) = 
    \begin{cases}
        \dfrac{\mathrm{count}(w_{i - 2} w_{i - 1} w_i)}{\mathrm{count}(w_{i - 2} w_{i - 1})} & \mathrm{count}(w_{i - 2} w_{i - 1}) > 0 \\
        0 & \text{otherwises.}
    \end{cases}\]

    模型求解可类比 1.3 中的二元情况,只须将 \(\mathrm{dp}\) 和 \(\mathrm{last}\) 拓展一维即可。

    \newpage

    \section{实验结果}

    下面测试使用的语料均为提供的 sina 新闻语料,测试输入文件为 \texttt{data/input.txt},结果文件为 \texttt{data/output.txt}。

    \subsection{准确率展示}

    字二元模型的准确率如下:

    \begin{table}[H]
        \centering
        \begin{tabular}{c|cc}
            \toprule
            \multicolumn{1}{c}{$\alpha$} & 整句正确率 & 逐字正确率 \\ \hline
            $0.9$ & $1.80\%$ & $55.01\%$ \\ \hline
            $0.99$ & $7.06\%$ & $64.44\%$ \\ \hline
            $0.999$ & $23.64\%$ & $77.54\%$ \\ \hline
            $0.9999$ & $36.28\%$ & $83.17\%$ \\ \hline
            $0.99999$ & $\pmb{39.08\%}$ & $\pmb{83.83\%}$ \\ \hline
            $0.999999$ & $38.91\%$ & $83.53\%$ \\ \hline
            $0.9999999$ & $38.58\%$ & $83.50\%$ \\
            \bottomrule
        \end{tabular}
        \caption{不同 $\alpha$ 值字二元模型表现}
	    \label{tab1}
    \end{table}

    字三元模型的准确率如下:

    \begin{table}[H]
        \centering
        \begin{tabular}{cccc}
            \toprule
            \multicolumn{2}{c}{参数} & \multicolumn{1}{c}{\multirow{2}{*}{整句正确率}} & \multirow{2}{*}{逐字正确率} \\
            $\alpha$ & \multicolumn{1}{c}{$\beta$} & \multicolumn{1}{c}{} & \multicolumn{1}{c}{} \\ \cline{1-4}
            \multicolumn{1}{c}{\multirow{4}{*}{$0.9999$}} & \multicolumn{1}{|c|}{$0.999$} & \multicolumn{1}{c}{$64.20\%$} & \multicolumn{1}{c}{$91.21\%$} \\ \cline{2-4}
            \multicolumn{1}{c}{} & \multicolumn{1}{|c|}{$0.99$} & \multicolumn{1}{c}{$65.02\%$} & \multicolumn{1}{c}{$91.38\%$} \\ \cline{2-4}
            \multicolumn{1}{c}{} & \multicolumn{1}{|c|}{$0.95$} & \multicolumn{1}{c}{$\pmb{65.51\%}$} & \multicolumn{1}{c}{$91.51\%$} \\ \cline{2-4}
            \multicolumn{1}{c}{} & \multicolumn{1}{|c|}{$0.9$} & \multicolumn{1}{c}{$65.18\%$} & \multicolumn{1}{c}{$91.74\%$} \\ \cline{1-4}
            \multicolumn{1}{c}{\multirow{4}{*}{$0.99999$}} & \multicolumn{1}{|c|}{$0.999$} & \multicolumn{1}{c}{$63.71\%$} & \multicolumn{1}{c}{$91.24\%$} \\ \cline{2-4}
            \multicolumn{1}{c}{} & \multicolumn{1}{|c|}{$0.99$} & \multicolumn{1}{c}{$64.53\%$} & \multicolumn{1}{c}{$91.41\%$} \\ \cline{2-4}
            \multicolumn{1}{c}{} & \multicolumn{1}{|c|}{$0.95$} & \multicolumn{1}{c}{$65.18\%$} & \multicolumn{1}{c}{$\pmb{91.76\%}$} \\ \cline{2-4}
            \multicolumn{1}{c}{} & \multicolumn{1}{|c|}{$0.9$} & \multicolumn{1}{c}{$65.18\%$} & \multicolumn{1}{c}{$91.74\%$} \\ \cline{1-4}
            \multicolumn{1}{c}{\multirow{4}{*}{$0.999999$}} & \multicolumn{1}{|c|}{$0.999$} & \multicolumn{1}{c}{$63.38\%$} & \multicolumn{1}{c}{$91.23\%$} \\ \cline{2-4}
            \multicolumn{1}{c}{} & \multicolumn{1}{|c|}{$0.99$} & \multicolumn{1}{c}{$64.20\%$} & \multicolumn{1}{c}{$91.36\%$} \\ \cline{2-4}
            \multicolumn{1}{c}{} & \multicolumn{1}{|c|}{$0.95$} & \multicolumn{1}{c}{$64.86\%$} & \multicolumn{1}{c}{$91.72\%$} \\ \cline{2-4}
            \multicolumn{1}{c}{} & \multicolumn{1}{|c|}{$0.9$} & \multicolumn{1}{c}{$64.36\%$} & \multicolumn{1}{c}{$91.67\%$} \\
            \bottomrule
        \end{tabular}
        \caption{不同 $\alpha, \beta$ 值字三元模型表现}
	    \label{tab2}
    \end{table}

    可见字三元模型相对字二元模型,无论是整句正确率还是逐字正确率都有了较明显的提升,特别是整句正确率。

    二元模型中,$\alpha$ 的值设置在 $0.999$ 以下时,效果较差,特别是对于整个句子的解析基本都不到位;而在 $0.9999$ 以上,效果基本接近,基本差别不大。

    三元模型中,取了二元模型表现较好的几个 $\alpha$ 值进行测试,可以发现 $\beta$ 值对于正确率的影响较小,基本在 $\beta = 0.95$ 左右时效果最佳。

    \subsection{正确案例}

    下面选取一些拼写正确的案例来证明输入法的效果,均为字三元模型的案例。

    \begin{itemize}
        \item \texttt{qing hua da xue shi shi jie yi liu da xue} \\
        清华大学是世界一流大学
        \item \texttt{ting che zuo ai feng lin wan} \\
        停车坐爱枫林晚
        \item \texttt{xun xun mi mi leng leng qing qing qi qi can can qi qi} \\
        寻寻觅觅冷冷清清凄凄惨惨戚戚
        \item \texttt{mei ge si nian yi ci de ao yun hui jiu yao zhao kai le} \\
        每隔四年一次的奥运会就要召开了
        \item \texttt{wei ji bai ke shi yi ge wang luo bai ke quan shu xiang mu} \\
        维基百科是一个网络百科全书项目
        \item \texttt{yi ge zi ren wei qian li bu fan de ren tong guo jian ku zhuo jue de nu li} \\
        一个自认为潜力不凡的人通过艰苦卓绝的努力
        \item \texttt{zhong guo pin kun di qu shi xian wang luo fu wu quan fu gai} \\
        中国贫困地区实现网络服务全覆盖
        \item \texttt{ben ci pu cha huo dong you zhu yu bang zhu tong xue men zou chu xin li wu qu} \\
        本次普查活动有助于帮助同学们走出心理误区
        \item \texttt{xiao chu kong ju de zui hao ban fa jiu shi mian dui kong ju} \\
        消除恐惧的最好办法就是面对恐惧
        \item \texttt{duo qu xin shi dai zhong guo te se she hui zhu yi wei da sheng li} \\
        夺取新时代中国特色社会主义伟大胜利
    \end{itemize}

    \subsection{错误案例}

    下面选取一些拼写错误的案例,均为字三元模型的案例。

    \begin{itemize}
        \item \texttt{yi zhi ke ai de da huang gou} \\
        \textbf{输出}:一只可爱的大\textbf{皇沟} \\
        \textbf{正解}:一只可爱的大\textbf{黄狗}
        \item \texttt{pin yin zhi jian yong kong ge ge kai} \\
        \textbf{输出}:\textbf{品音}之间用空格隔开 \\
        \textbf{正解}:\textbf{拼音}之间用空格隔开
        \item \texttt{kai tong jin jin si shi ba xiao shi xi fen er shi jiu wan} \\
        \textbf{输出}:开通仅仅四十八小时\textbf{细分}二十九万 \\
        \textbf{正解}:开通仅仅四十八小时\textbf{吸粉}二十九万
        \item \texttt{ni jia wo de wei xin} \\
        \textbf{输出}:你\textbf{枷}我的微信 \\
        \textbf{正解}:你\textbf{加}我的微信
        \item \texttt{hua wei dui dai cheng xu yuan ru he} \\
        \textbf{输出}:\textbf{化为}对待程序员如何 \\
        \textbf{正解}:\textbf{华为}对待程序员如何
    \end{itemize}

    \subsection{案例分析}

    根据正确案例,可以看到,对于长难句,乃至一些简单的古文,输入法都能给出准确的回答。

    但错误案例中,也有一些较为简单的句子未能给出一个正确的回答,甚至给出的回答是不太合理的。
    
    这一方面体现的是语料库本身可能有一些侧重点的问题,比如\textbf{黄狗}判断成\textbf{皇沟}可能与语料库中较少有“大黄狗”等表达有关。
    
    另一方面,输入法对于较长的上下文缺乏判断能力,比如\textbf{加}我的微信被判断为了\textbf{枷},以及\textbf{华为}判断为\textbf{化为},这是字三元模型算法本身的一个弊端。

    总结而言,模型的弊端主要分为\textbf{算法本身的局限}和\textbf{语料库的偏向性}。对于后者,一些更大更全面的数据集或语料库可能可以较好地解决;对于前者,则可能需要加入词多元模型或者 NLP 模型,使得模型可以更深入理解句子本身。

    \section{运行方法与环境}

    \subsection{环境要求}

    除用于输入输出分离的源文件 \texttt{process.py} 位于 \texttt{data} 文件夹下外,所有的 \texttt{.py} 源文件均位于 \texttt{src} 文件夹下。
    
    Python 环境建议为 Python 3.8,主要用到的 Python 库如下(可参考\texttt{requirements.txt}文件):
    \begin{itemize}
        \item \texttt{tqdm} 4.59.0
        \item \texttt{pypinyin} 0.41.0
        \item \texttt{jieba} 0.42.1
    \end{itemize}

    其余所用库均为 Python 自带库。

    \subsection{使用方法}

    模型在命令行中使用:

    \begin{itemize}
        \item \texttt{usage: python -m src.main [-h] [--model MODEL] [--input INPUT]} \\
        \texttt{[--output OUTPUT] [--nocheck] [--answer ANSWER]}
        \item \texttt{-m, --model} 指定模型名称,可选 \texttt{BinaryModel} 和 \texttt{TernaryModel},默认为 \texttt{TernaryModel}。
        \item \texttt{-i, --input} 指定输入文件,默认为 \texttt{data/input.txt}。
        \item \texttt{-o, --output} 指定输出文件,默认为 \texttt{data/test.txt}。
        \item \texttt{-a, --answer} 指定答案文件,默认为 \texttt{data/output.txt}。
        \item \texttt{-n, --nocheck} 指定是否比对答案,默认为 $0$,即比对。
    \end{itemize}

    \section{总结}

    本次输入法实验是我第一次做一个有实践意义的人工智能实验,这次实验让我对于人工智能领域中的一些概念,比如评价函数、预测模型、参数设置等有了更深入的了解。

    通过实验,我获得了以下结论:

    \begin{itemize}
        \item 更多元的模型会有更好的效果,但同时也会带来更大的性能与负载压力。
        \item 不同的参数设置在不同模型中取得的效果可能会有较大差别,对于参数的调整是十分有必要的。
        \item 语料库对于模型的建立十分重要,不同的语料库在相同的测试下可能导致较大的结果差异。
    \end{itemize}

    除此之外,通过实验与数据分析,我也思考了模型的几个可能的改进方向:

    \begin{itemize}
        \item 尝试利用分词来增加词的二元乃至三元转移,可能可以对于一些出现频率较低的词汇有更好的效果。
        \item 增加更多的语料库,比如增加 bilibili、知乎等平台的语料,有助于加强对于一些新兴词汇的解析。
        \item 考虑引入神经网络等结构,通过一些 NLP 的模型来加入语义理解等。
    \end{itemize}

\end{document}
